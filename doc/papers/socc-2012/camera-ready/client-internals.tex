\subsection{Subscription Client Internals}
The Databus subscription client is the glue between the Databus infrastructure (relays, bootstrap service) 
and the Application (business logic in the consumer). 
The client is responsible for connecting to the appropriate relay and bootstrap service clusters, keeping track of progress in the Databus event stream, switching over automatically between the Relays and Bootstrap service when necessary, and performing conversions between the schema used for serialization of the payload and the schema expected by the consumer. 

The client runs a fetcher thread that is pulling continuously from the relay over a persistent HTTP connection and a dispatcher thread that fires callbacks into a pool of worker threads. There is local flow control between the fetcher and the dispatcher to ensure a steady stream of Databus events to the consumer. Two forms of multi-threaded processing are supported. The first type allows multi-threaded processing within a consistency window only. This ensures that consistency semantics are maintained at the destination. The second type allows multi-threaded processing without regard to the transaction window boundaries. This provides higher throughput at the consumer at the cost of relaxed consistency. 
The client maintains state of where it is in the sequence timeline through a customizable CheckpointPersister. By default, a checkpoint is persisted to disk for every successfully-processed consistency window. Applications that need very close control of the checkpoint may implement their own storage and restore for checkpoints to tie it to their processing state. For example, search indexes will often commit the checkpoint as meta-data along with the index files themselves so that they share the same fate. 

