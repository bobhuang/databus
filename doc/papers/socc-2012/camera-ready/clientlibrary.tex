\subsection{Event Model and Consumer API}

\lstset{basicstyle=\small}


There are two versions of the consumer API, one that is callback driven and another that is iterator-based. 
At a high-level, there are eight main methods on the Databus callback API.
%%We show the callback based API at Listing~\ref{listing:DatabusConsumerAPI}.   
%%\begin{algorithm}
%%\lstset{caption={Databus Consumer API},label=listing:DatabusConsumerAPI}
%%\begin{lstlisting}
%%interface DatabusEventListener
%%{
%%  Result onStartDataEventSequence(SCN startScn);
%% Result onEndDataEventSequence(SCN endScn);
%% Result onStartSource(String source, 
%%                       Schema srcSchema);
%%  Result onEndSource(String source, 
%%                     Schema srcSchema);
%%  Result onDataEvent(DbusEvent e, 
%%                     DbusEventDecoder decoder);
%%  Result onCheckpoint(SCN checkpointScn);
%%  Result onRollback(SCN rollbackScn);
%%  Result onError(SCN rollbackScn);
%%}
%%\end{lstlisting}
%%\end{algorithm}

\begin{itemize*}
\item \emph{onStartDataEventSequence}: the start of a sequence of data events from an events consistency window.
\item \emph{onStartSource}: the start of data events from the same Databus source (e.g. Oracle table). 
\item \emph{onDataEvent}: a data change event for the current Databus source.
\item \emph{onEndSource}: the end of data change events from the same Databus source.
\item \emph{onEndDataEventSequence}: the end of a sequence of data events with the same SCN.
\item \emph{onCheckpoint}: a hint from the Databus client library that it wants to mark the point in the stream identified by the SCN as a recovery point
\item \emph{onRollback}: Databus has detected a recoverable error while processing the current event consistency window and it will rollback to the last successful checkpoint.
\item \emph{onError}: Databus has detected a unrecoverable error and it will stop processing the event stream.
\end{itemize*}.

The above callbacks denote the important points in the stream of Databus change events. A typical sequence of callbacks follows the pattern below.

\begin{verbatim}
onStartDataEventSequence(startSCN)
    onStartSource(Table1)
        onDataEvent(Table1.event1)
               ...
        onDataEvent(Table1.eventN) 
    onEndSource(Table1)
    onStartSource(Table2)
        onDataEvent(Table2.event1) 
              ...
        onDataEvent(Table2.eventM)
    onEndSource(Table2)
        ... 
onEndDataEventSequence(endSCN)
\end{verbatim}

Intuitively, the Databus client communicates with the consumer: ``Here is the next batch of changes in the watched tables (sources). The changes are broken down by tables. Here are the changes in the first table, then the changes to the next table, etc. All the changes represent the delta from the previous consistent state of the database to the following consistent state.''

The contract on all of the callbacks is that the processing code can return a result code denoting a successful processing of the callback, recoverable or unrecoverable error. Failures to process the callback within the allocated time budget or throwing an exception, result in a recoverable error.
In cases of recoverable errors, the client library will rollback to the last successful checkpoint and replay the callbacks from that point.

The offloading of state-keeping responsibility from the consumer simplifies the consumer recovery. The consumer or a newly spawned consumer can rewind back to the last known good checkpoint. For instance, if the consumer is stateful, they just need to tie the state that they are keeping with the checkpoint of the stream. On failure, the new consumer can read the state and the checkpoint associated with it. With this, it may start consuming from that point. If the stream consumption is idempotent, then the checkpoint can be maintained lazily as well. 

\subsection{Partitioned Stream Consumption}
Databus supports independent partitioning of both the data source and the consumers. This allows independent scaling of the data source tier from the processing tier. 
For example, we have monolithic data sources which emit a single stream, but are consumed by partitioned search indexes. Conversely, there are partitioned data stores which are consumed as a single logical stream by downstream consumers. The partitioning cardinality can change over time on either side without affecting the other side. 
In the case of partitioned data sources, Databus currently enforces transactional semantics only at the partition level, and provides in-order and at-least once guarantees of delivery per source partition. 


%We call this M:N partitioning, signifying that the source could be partitioned 1 or M ways and the consumer could consume it partitioned 1 or N ways. 
%It supports un-partitioned streams being subscribed by partition-aware consumers 
%Why is this important? Databases are often partitioned horizontally for scalability. 
%As the read and write load changes, databases will often get re-partitioned repeatedly. 

There are three primary categories of partitioning scenarios:
\begin{enumerate}
\item\emph{Single consumer}: A consumer that is subscribing to the change stream from a logical database must be able to do so independently of the physical partitioning of the database. This is supported by just doing a merge of the streams emanating from the source database partitions.
\item\emph{Partition-aware consumer}: The consumer may have affinity to certain partitions of the stream and want to process only events that affect those partitions. This form of partitioning is implemented through server-side filters and includes mod-based, range-based and hash-based partitioning.  This filtering is natively supported in the Relay and the Bootstrap Service.  The consumers specify the partitioning scheme to be used as part of their registration in the Subscription Client. The Subscription Client then includes it as part of the pull requests to the Relay or the Bootstrap Service. The Bootstrap Service may apply further optimization like predicate push-down so that only the necessary events are read from the persistent log or snapshot stores.
\item\emph{Consumer groups}: The change stream may be too fast for a single consumer to process and the processing needs to be scaled out across multiple physical consumers who act as a logical group. This is implemented by using a generic cluster management framework called Helix. The partitions of the database are assigned to the consumers in the same group so that every partition has exactly R consumers in the group assigned to it and the partitions are evenly distributed among the consumers. When any consumer fails, Helix rebalances the assignment of partitions by moving the partitions assigned to the failed consumer to the surviving consumers. If the existing consumers in the group are not able to keep up with the stream, additional consumers might be added to the group to scale the processing. In this case, the assignment of partitions is rebalanced so that some partitions from each existing consumer are moved to the new consumers. 
%%Once partitions are assigned, the consumers just register using partition-specific filters to consume the change stream. 
\end{enumerate}

%A consumer that is subscribing to the change stream from a logical database must be able to do so independent of the physical partitioning of the database. 
%Secondly, often the change stream will be too fast for a single consumer to process and the processing needs to be scaled out across multiple physical consumers who act as a logical group. 

%Partitioning is implemented through server-side filters and includes mod-based, range-based and hashed-based partitioning.  This filtering is natively supported in the Relay and the Bootstrap Service.  The consumers specify the partitioning scheme to be used as part of their registration in the Subscription Client. The Subscription Client then includes it as part of the pull requests to the Relay or the Bootstrap Service. The Bootstrap Service may apply further optimization like predicate push-down so that only the necessary events are read from the persistent log or snapshot stores.

%Databus also implements the notion of a consumer group, using the generic cluster manager Helix. The partitions of the database are assigned to the consumers in the same group so that every partition has one and exactly one consumer in the group assigned to it and the partitions are evenly distributed among the consumers. When any consumer fails, Helix rebalances the assignment of partitions by moving the partitions assigned to the failed consumer to the surviving consumers. If the existing consumers in the group are not able to keep up with the stream, additional consumers might be added to the group to scale the processing. In this case, the assignment of partitions is rebalanced so that some partitions from each existing consumer are moved to the new consumers.



